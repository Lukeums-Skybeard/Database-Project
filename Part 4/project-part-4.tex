%!TEX encoding = UTF-8 Unicode
%!TEX TS-program = lualatexmk-shell
\RequirePackage{ucharcat,uspace,etex,shellesc}
\RequirePackage[1.7,compress,objcompress]{bxpdfver}
\documentclass[american,extrafontsizes,12pt,portrait,letterpaper,oneside,onecolumn,article,final]{memoir}
\settypeoutlayoutunit{in}
\pagestyle{plain}
\setulmarginsandblock{1in}{1in}{*}
\setlrmarginsandblock{1in}{1in}{*}
\setheadfoot{0.5in}{0.5in}
\usepackage{etoolbox,indentfirst,pdftexcmds,ragged2e,magicnum,atveryend,regexpatch,graphicx,dotlessi,checkend,calc,alphalph,thinsp,pbox,xfrac,xsavebox,xspace}
\usepackage[usestackEOL]{stackengine}
\usepackage[table]{xcolor}
\usepackage{colorspace,longtable}
\usepackage[longtable]{multirow}
\usepackage{bigstrut,bigdelim,fontaxes,arydshln,diagbox,collcell,scalerel,hhline,fvextra}
\usepackage{amssymb,amsmath,amsthm,delim}
\usepackage{accents,mathtools,thmtools,empheq,iffont,unicode-math,realscripts,metalogo,mathpartir,luatextra,minted}
\renewcommand\MintedPygmentize{/Library/Frameworks/Python.framework/Versions/Current/bin/pygmentize}
\usepackage[normalem]{ulem}
\setmainfont{DejaVuSerif}[
Path={/Library/Fonts/},
Extension={.ttf},
Scale=MatchLowercase,
UprightFont={*},
BoldFont={*-Bold},
ItalicFont={*-Italic},
BoldItalicFont={*-BoldItalic},
FontFace={c}{n}{*Condensed},
FontFace={cb}{n}{*Condensed-Bold},
FontFace={c}{it}{*Condensed-Italic},
FontFace={cb}{it}{*Condensed-BoldItalic}
]
\setsansfont{DejaVuSans}[
Path={/Library/Fonts/},
Extension={.ttf},
Scale=MatchLowercase,
UprightFont={*},
BoldFont={*-Bold},
ItalicFont={*-Oblique},
BoldItalicFont={*-BoldOblique},
FontFace={c}{n}{*Condensed},
FontFace={cb}{n}{*Condensed-Bold},
FontFace={c}{it}{*Condensed-Oblique},
FontFace={cb}{it}{*Condensed-BoldOblique},
FontFace={el}{n}{*-ExtraLight}
]
\setmonofont{DejaVuSansMono}[
Path={/Library/Fonts/},
Extension={.ttf},
Scale=MatchLowercase,
UprightFont={*},
BoldFont={*-Bold},
ItalicFont={*-Oblique},
BoldItalicFont={*-BoldOblique}
]
\setmathfont{DejaVuSerif}[
Path={/Library/Fonts/},
Extension={.ttf},
Scale=MatchLowercase,
range={up->up},
script-features={},
sscript-features={}
]
\setmathfont{DejaVuSerif-Bold}[
Path={/Library/Fonts/},
Extension={.ttf},
Scale=MatchLowercase,
range={bfup->up},
script-features={},
sscript-features={}
]
\setmathfont{DejaVuSerif-Italic}[
Path={/Library/Fonts/},
Extension={.ttf},
Scale=MatchLowercase,
range={it->up},
script-features={},
sscript-features={}
]
\setmathfont{DejaVuSerif-BoldItalic}[
Path={/Library/Fonts/},
Extension={.ttf},
Scale=MatchLowercase,
range={bfit->up},
script-features={},
sscript-features={}
]
\setmathfont{DejaVuSans}[
Path={/Library/Fonts/},
Extension={.ttf},
Scale=MatchLowercase,
range={sfup->up,bb->bb,bbit->bbit},
script-features={},
sscript-features={}
]
\setmathfont{DejaVuSans-Bold}[
Path={/Library/Fonts/},
Extension={.ttf},
Scale=MatchLowercase,
range={bfsfup->up},
script-features={},
sscript-features={}
]
\setmathfont{DejaVuSans-Oblique}[
Path={/Library/Fonts/},
Extension={.ttf},
Scale=MatchLowercase,
range={sfit->up},
script-features={},
sscript-features={}
]
\setmathfont{DejaVuSans-BoldOblique}[
Path={/Library/Fonts/},
Extension={.ttf},
Scale=MatchLowercase,
range={bfsfit->up},
script-features={},
sscript-features={}
]
\setmathfont{DejaVuSansMono}[
Path={/Library/Fonts/},
Extension={.ttf},
Scale=MatchLowercase,
range={tt->up},
script-features={},
sscript-features={}
]
\setmathfont{DejaVuMathTeXGyre}[
Path={/Library/Fonts/},
Extension={.ttf},
Scale=MatchLowercase
]
\usepackage[inline]{enumitem}
\usepackage{linegoal}
\usepackage[linegoal,delarray]{tabu}
\usepackage{hyphenat,hyperxmp,hyperref,polyglossia}
\setdefaultlanguage[variant=american]{english}
\usepackage[english=american,maxlevel=6]{csquotes}
\usepackage[american]{selnolig}
\usepackage[calc,useregional]{datetime2}
\usepackage{microtype,tikz,tikzpagenodes}
\usetikzlibrary{babel,scopes,quotes,calc,through,fit,matrix,chains,arrows.meta,circuits.logic.US,graphs,graphdrawing,automata,positioning,er,tikzmark}
\usegdlibrary{trees,layered,force,circular,routing}
\usepackage[tikz]{ocgx2}
%\usepackage{showhyphens}
\microtypesetup{activate=all}
\hypersetup{unicode=true,pdflang={en}}
\usemintedstyle{xcode}
\nolig{ffic}{f|fic}
\nolig{affin}{af|fin}
\tikzstyle{every picture}+=[remember picture,outer sep=auto]

\newcommand*\rnmath[1]{\(\displaystyle #1 \textstyle\)}
\newcommand\band{\mathbin{∧}}
\newcommand\bor{\mathbin{∨}}
\newcommand\bnot{\mathopen{∼}}
\newcommand\TT{\mathord{\symsfup{T}}}
\newcommand\FF{\mathord{\symsfup{F}}}
\newcommand\taut{\mathord{\symbfup{t}}}
\newcommand\cont{\mathord{\symbfup{c}}}
\newcommand\bqed{\mathpunct{∎}}
\newcommand\blank{\mathord{-}}
\newcommand\divslash{\mathbin{∕}}
\newcommand\fracslash{\mathbin{⁄}}
\DeclarePairedDelimiterX\divides[2]{}{}{#1 \mathrel{\delimsize∣} #2}
\DeclarePairedDelimiterX\ndivides[2]{}{}{#1 \mathrel{\delimsize∤} #2}
\newcommand\negative{\mathopen{-}}
\newcommand\mult{\mathbin{·}}
\newcommand\factorial{\mathclose{!}}
\DeclarePairedDelimiter\floor{⌊}{⌋}
\DeclarePairedDelimiter\ceiling{⌈}{⌉}
\newcommand\exto{\mathpunct{϶}}
\DeclarePairedDelimiter\card{\lvert}{\rvert}
\newcommand*\rproj[1]{\mathop{π_{#1}}}
\newcommand*\rsel[1]{\mathop{σ_{#1}}}
\newcommand*\rname[1]{\mathop{ρ_{#1}}}
\newcommand*\rjoin[1]{\mathbin{⋈_{#1}}}
\newcommand\rcross{\mathbin{×}}
\newcommand\runion{\mathbin{∪}}
\newcommand\rintersection{\mathbin{∩}}
\newcommand\rdiff{\mathbin{-}}
\colorlet{gRed}{red!75!black}
\colorlet{gBlue}{blue!75!black}
\colorlet{gGray}{white!25!black}
\newcommand*\email[2][]{\href[#1]{mailto:#2}{\nolinkurl{#2}}}
\newcommand*\TODO{\textcolor{red}{[TODO]}\xspace}
\newcommand*{\sqli}[1]{\mintinline[breaklines,breakbytoken]{sql}{#1}}
\newcommand*{\sqlm}[1]{\text{\sqli{#1}}}

\author{Timothy Gibson, Alexander Altman, and Schuyler Davis}
\title{Team GLASTA's \emph{Fantastic Furniture}:\\*Normalization}
\date{\DTMdate{2017-04-13}}

\begin{document}
\checkandfixthelayout[nearest]
\midsloppy
\begin{Center}
\textsf{\strong{\LARGE\thetitle}}
\end{Center}

\begin{samepage}%
\section*{Team Info}
\pdfbookmark[0]{Team Info}{team-info}%
\begin{Center}
\begin{tabu}[t]{rlr}
\strong{Team Name:} & \multicolumn{2}{c}{Team GLASTA}\\
\tabucline[2\arrayrulewidth]{-}
\strong{Project Name:} & \multicolumn{2}{c}{\emph{Fantastic Furniture}}\\
\tabucline[2\arrayrulewidth]{-}
\ldelim\{{3}{*}[\strong{Participants:}] & Timothy Gibson & \email{tgibson1@csustan.edu}\\
\tabucline{2-}
& Alexander Altman & \email{aaltman@csustan.edu}\\
\tabucline{2-}
& Schuyler Davis & \email{sdavis20@csustan.edu}\\
\end{tabu}
\end{Center}%
\end{samepage}

\section*{Initial Relations}
\pdfbookmark[0]{Initial Relations}{original}%

\inputminted[linenos]{sql}{original.sql}

\subsection*{Functional Dependencies}
\pdfbookmark[1]{Functional Dependencies}{fundeps}%

\begin{longtabu}{rcll}
\sqli{supplierID} & \rnmath{\longrightarrow} & \sqli{name_} & \rdelim\}{5}{*}[\sqli{Supplier}]\\*
\sqli{supplierID} & \rnmath{\longrightarrow} & \sqli{phone} &\\*
\sqli{supplierID} & \rnmath{\longrightarrow} & \sqli{address} &\\*
\sqli{supplierID} & \rnmath{\longrightarrow} & \sqli{country} &\\*
\sqli{supplierID} & \rnmath{\longrightarrow} & \sqli{website} &\\
\sqli{designerID} & \rnmath{\longrightarrow} & \sqli{name_} & \rdelim\}{6}{*}[\sqli{Designer}]\\*
\sqli{designerID} & \rnmath{\longrightarrow} & \sqli{phone} &\\*
\sqli{designerID} & \rnmath{\longrightarrow} & \sqli{address} &\\*
\sqli{designerID} & \rnmath{\longrightarrow} & \sqli{country} &\\*
\sqli{designerID} & \rnmath{\longrightarrow} & \sqli{website} &\\*
\sqli{designerID} & \rnmath{\longrightarrow} & \sqli{designFocus} &\\
\sqli{setID} & \rnmath{\longrightarrow} & \sqli{name_} & \rdelim\}{4}{*}[\sqli{Set_}]\\*
\sqli{setID} & \rnmath{\longrightarrow} & \sqli{catalogYear} &\\*
\sqli{setID} & \rnmath{\longrightarrow} & \sqli{catalogNumber} &\\*
\sqli{setID} & \rnmath{\longrightarrow} & \sqli{style_} &\\
\sqli{modelNumber} & \rnmath{\longrightarrow} & \sqli{name_} & \rdelim\}{5}{*}[\sqli{Model}]\\*
\sqli{modelNumber} & \rnmath{\longrightarrow} & \sqli{material} &\\*
\sqli{modelNumber} & \rnmath{\longrightarrow} & \sqli{upholstery} &\\*
\sqli{modelNumber} & \rnmath{\longrightarrow} & \sqli{durability} &\\*
\sqli{modelNumber} & \rnmath{\longrightarrow} & \sqli{color} &\\
\sqli{sku} & \rnmath{\longrightarrow} & \sqli{name_} & \rdelim\}{6}{*}[\sqli{Item}]\\*
\sqli{sku} & \rnmath{\longrightarrow} & \sqli{dimensions.length_} &\\*
\sqli{sku} & \rnmath{\longrightarrow} & \sqli{dimensions.width} &\\*
\sqli{sku} & \rnmath{\longrightarrow} & \sqli{dimensions.height} &\\*
\sqli{sku} & \rnmath{\longrightarrow} & \sqli{condition} &\\*
\sqli{sku} & \rnmath{\longrightarrow} & \sqli{weightLimit} &\\
\sqli{centerID} & \rnmath{\longrightarrow} & \sqli{name_} & \rdelim\}{5}{*}[\sqli{DistributionCenter}]\\*
\sqli{centerID} & \rnmath{\longrightarrow} & \sqli{phone} &\\*
\sqli{centerID} & \rnmath{\longrightarrow} & \sqli{address} &\\*
\sqli{centerID} & \rnmath{\longrightarrow} & \sqli{country} &\\*
\sqli{centerID} & \rnmath{\longrightarrow} & \sqli{website} &\\
\multicolumn{4}{c}{\emph{Note: the \sqli{make} relation has no nontrivial functional dependencies.}}\\
\rnmath{\{\sqlm{setID}, \sqlm{modelNumber}\}} & \rnmath{\longrightarrow} & \sqli{count_} & \rdelim\}{1}{*}[\sqli{contains_}]\\
\sqli{sku} & \rnmath{\longrightarrow} & \sqli{modelNumber} & \rdelim\}{1}{*}[\sqli{describes}]\\
\rnmath{\{\sqlm{centerID}, \sqlm{supplierID}\}} & \rnmath{\longrightarrow} & \sqli{leadTime} & \rdelim\}{1}{*}[\sqli{canOrderFrom}]\\
\sqli{sku} & \rnmath{\longrightarrow} & \sqli{centerID} & \rdelim\}{1}{*}[\sqli{stocks}]\\
\sqli{sku} & \rnmath{\longrightarrow} & \sqli{numberOfLegs} & \rdelim\}{5}{*}[\sqli{Chair}]\\*
\sqli{sku} & \rnmath{\longrightarrow} & \sqli{hasCushion} &\\*
\sqli{sku} & \rnmath{\longrightarrow} & \sqli{hasArms} &\\*
\sqli{sku} & \rnmath{\longrightarrow} & \sqli{backHeight} &\\*
\sqli{sku} & \rnmath{\longrightarrow} & \sqli{seatHeight} &\\
\sqli{sku} & \rnmath{\longrightarrow} & \sqli{numberOfLegs} & \rdelim\}{3}{*}[\sqli{Table_}]\\*
\sqli{sku} & \rnmath{\longrightarrow} & \sqli{numberOfSeats} &\\*
\sqli{sku} & \rnmath{\longrightarrow} & \sqli{shape} &\\
\sqli{sku} & \rnmath{\longrightarrow} & \sqli{angle} & \rdelim\}{2}{*}[\sqli{Desk}]\\*
\sqli{sku} & \rnmath{\longrightarrow} & \sqli{numberOfDrawers} &\\
\sqli{sku} & \rnmath{\longrightarrow} & \sqli{numberOfLegs} & \rdelim\}{3}{*}[\sqli{Stool}]\\*
\sqli{sku} & \rnmath{\longrightarrow} & \sqli{hasCushion} &\\*
\sqli{sku} & \rnmath{\longrightarrow} & \sqli{hasSwivel} &\\
\sqli{sku} & \rnmath{\longrightarrow} & \sqli{numberOfCompartments} & \rdelim\}{2}{*}[\sqli{Cabinet}]\\*
\sqli{sku} & \rnmath{\longrightarrow} & \sqli{capacity} &\\
\sqli{sku} & \rnmath{\longrightarrow} & \sqli{size_} & \rdelim\}{2}{*}[\sqli{Bedframe}]\\*
\sqli{sku} & \rnmath{\longrightarrow} & \sqli{depth_} &\\
\rnmath{\{\sqlm{modelNumber}, \sqlm{description}\}} & \rnmath{\longrightarrow} & \sqli{count_} & \rdelim\}{1}{*}[\sqli{features_Feature}]\\
\end{longtabu}

\subsubsection*{Functional Dependency Notes}
\pdfbookmark[2]{Functional Dependency Notes}{fundep-notes}%

One might think, at first, that several attributes of \sqli{Supplier} (such as \sqli{website} and \sqli{phone}) should be candidate keys by virtue of uniquely determining the primary key \sqli{supplierID}.
However, consider the following scenario: a supplier is an independent carpenter living in a country whose laws (for whatever reason) disallow any one single business from selling both chairs and bedframes.
This carpenter produces both types of items, but, because of the laws of his home country, he has to run two separate businesses from the legal perspective.
The inevitable result of us getting furniture from this carpenter is two \sqli{Supplier} tuples with (necessarily) distinct \sqli{supplierID}s but where \emph{every other attribute is identical}!
Therefore, by constructed counterexample, \sqli{Supplier}'s only candidate key is its primary key \sqli{supplierID}; very similar reasoning applies to \sqli{DistributionCenter} and \sqli{Designer} as well.

\section*{Normalized Relations}
\pdfbookmark[0]{Normalized Relations}{normalized}%

\inputminted[linenos]{sql}{normalized.sql}

\subsection*{Normal Forms}
\pdfbookmark[1]{Normal Forms}{normal-forms}%

\begin{enumerate}[leftmargin=*,widest={\texttt{DistributionCenter}}]

\item[\sqli{Supplier}]
\sqli{Supplier} is in BCNF.
Each functional dependency in the relation points back to \sqli{supplierID}.
Every other attribute of the relation is dependent on \sqli{supplierID} and as such is a member of the candidate key.
Also, every attribute of the \sqli{Supplier} relation is functionally determined by \sqli{supplierID}, making it the superkey.

\item[\sqli{Designer}]
\sqli{Designer} is in BCNF.
\sqli{designerID} determines every attribute of the \sqli{Designer} relation.
This effectively makes \sqli{designerID} the superkey for the \sqli{Designer} relation.

\item[\sqli{Set_}]
The \sqli{Set_} relation is in BCNF.
\sqli{Set_} has a candidate key that is \sqli{setID}.
Each attribute of \sqli{Set_} is unique to a specific \sqli{setID}.
This indicates that each attribute is determined by that setID.
This makes setID the superkey for \sqli{Set_}.
From this, it lends that \sqli{Set_} is in BCNF because that for every non\hyp trivial functional dependency, \sqli{setID} is the superkey.

\item[\sqli{Model}]
\sqli{Model} is in BCNF.
For every attribute of the \sqli{Model} relation, they are functionally determined by the \sqli{modelNumber}, meaning that they are only determined by one specific \sqli{modelNumber}.
This makes \sqli{modelNumber} the superkey for \sqli{Model}.

\item[\sqli{Item}]
The \sqli{Item} relation is in BCNF.
The attribute \sqli{sku} is a single identifier for each individual item.
Each item has one \sqli{name}, \sqli{length_}, \sqli{width}, \sqli{height}, and \sqli{weightLimit}. Each of these attributes are dependent on one \sqli{sku}.
This makes \sqli{sku} the superkey for \sqli{Item}.
Since all attributes are dependent on a single \sqli{sku} then \sqli{Item} is in BCNF.

In our original SQL, the \sqli{Item} relation used a type (\sqli{dimensions}) that was created specifically for that relation in its table.
That type's fields have now been defined inline inside of the table so that it complies with the requirements of BCNF.

\item[\sqli{DistributionCenter}]
\sqli{DistributionCenter} is in BCNF.
Each \sqli{name_}, \sqli{phone}, \sqli{address}, \sqli{country}, and \sqli{website} is specific to one \sqli{centerID}.
This makes \sqli{centerID} the superkey for the \sqli{DistributionCenter} relation.
Since each attribute is only populated by one tuple and the superkey determines every attribute, then the \sqli{DistributionCenter} relation is in BCNF.

\item[\sqli{make}]
The \sqli{make} relation is in BCNF, since it only contains trivial functional dependencies.

\item[\sqli{contains_}]
The \sqli{contains_} relation is in BCNF.
The \sqli{contains_} relation has a primary key of \rnmath{\{\sqlm{setID}, \sqli{modelNumber}\}}.
These two together effectively become the superkey and since the count of the \sqli{contains_} relation can be determined by the \sqli{setID} and \sqli{modelNumber}, \sqli{contains_} is in BCNF.

\item[\sqli{describes}]
The \sqli{describes} relation is in BCNF.
The \sqli{describes} relation includes two foreign keys that together also form the primary key.
These two keys are \sqli{modelNumber} and \sqli{sku}.
The \sqli{modelNumber} specifically determines a single \sqli{sku}.
This means that \sqli{sku} is dependent on the \sqli{modelNumber} and that \sqli{modelNumber} is the superkey. Since the only non\hyp trivial functional dependency in the \sqli{describes} relation involves the superkey determining the single other attribute, \sqli{describes} is in BCNF.

\item[\sqli{canOrderFrom}]
The \sqli{canOrderFrom} relation is in BCNF.
The \sqli{canOrderFrom} relation has an attribute called \sqli{leadTime}.
This \sqli{leadTime} is dependent on the distribution center (identified by \sqli{centerID}) and the supplier (identified by \sqli{supplierID}).
This \sqli{centerID} and \sqli{supplierID} are both part of the primary key for \sqli{canOrderFrom} and together make up its superkey.
For this reason, \sqli{canOrderFrom} is in BCNF.

\item[\sqli{stocks}]
The \sqli{stocks} relation is in BCNF.
The \sqli{stocks} relation has a primary key of \sqli{sku}.
Since \sqli{centerID} is dependent on \sqli{sku} and the \sqli{stocks} relation borrows both \sqli{sku} and \sqli{centerID} from other relations, \sqli{sku} is the superkey.
For this reason, \sqli{stocks} is in BCNF.

\item[\sqli{Chair}]
The \sqli{Chair} relation is in BCNF.
The \sqli{Chair} relation is a subset of the \sqli{Item} relation.
Each chair has one of each of its attributes that is strictly related to its \sqli{sku}.
This makes the \sqli{sku} the superkey for the \sqli{Chair} relation.
As such, since each functional dependency is dependent on \sqli{sku}, \sqli{Chair} is in BCNF.

\item[\sqli{Table_}]
The \sqli{Table_} relation is in BCNF.
The \sqli{Table_} relation is a subset of the \sqli{Item} relation.
Each table has one of each of its attributes that is strictly related to its \sqli{sku}.
This makes the \sqli{sku} the superkey for the \sqli{Table_} relation.
As such, since each functional dependency is dependent on \sqli{sku}, \sqli{Table_} is in BCNF.

\item[\sqli{Desk}]
The \sqli{Desk} relation is in BCNF.
The \sqli{Desk} relation is a subset of the \sqli{Item} relation.
Each desk has one of each of its attributes that is strictly related to its \sqli{sku}.
This makes the \sqli{sku} the superkey for the \sqli{Desk} relation.
As such, since each functional dependency is dependent on \sqli{sku}, \sqli{Desk} is in BCNF.


\item[\sqli{Stool}]
The \sqli{Stool} relation is in BCNF.
The \sqli{Stool} relation is a subset of the \sqli{Item} relation.
Each stool has one of each of its attributes that is strictly related to its \sqli{sku}.
This makes the \sqli{sku} the superkey for the \sqli{Stool} relation.
As such, since each functional dependency is dependent on \sqli{sku}, \sqli{Stool} is in BCNF.

\item[\sqli{Cabinet}]
The \sqli{Cabinet} relation is in BCNF.
The \sqli{Cabinet} relation is a subset of the \sqli{Item} relation.
Each cabinet has one of each of its attributes that is strictly related to its \sqli{sku}.
This makes the \sqli{sku} the superkey for the \sqli{Cabinet} relation.
As such, since each functional dependency is dependent on \sqli{sku}, \sqli{Cabinet} is in BCNF.


\item[\sqli{Bedframe}]
The \sqli{Bedframe} relation is in BCNF.
The \sqli{Bedframe} relation is a subset of the Item relation.
Each bedframe has one of each of its attributes that is strictly related to its \sqli{sku}.
This makes the \sqli{sku} the superkey for the \sqli{Bedframe} relation.
As such, since each functional dependency is dependent on \sqli{sku}, \sqli{Bedframe} is in BCNF.

\item[\sqli{features_Feature}]
This relation is in BCNF.
The \sqli{features_Feature} relation has a key that it contains called \sqli{description}.
This \sqli{description}, however long it may be, will be distinct meaning that each feature has a specific \sqli{description} fitting to that specific feature.
This makes \rnmath{\{\sqlm{modelNumber}, \sqlm{description}\}} the superkey for \sqli{features_Feature}.
Also, since \sqli{count_} is dependent specifically on the description of that feature for that model, it is dependent on the superkey; this means that \sqli{features_Feature} is in BCNF.

\end{enumerate}

\section*{Group Work}
\pdfbookmark[0]{Group Work}{group-work}%
\begin{samepage}%
\begin{enumerate}[leftmargin=*,widest={Alexander:}]

\item[Alexander:]
\pdfbookmark[1]{Alexander}{Alexander}%
Modified the SQL to 1NF; looked over the conversions to BCNF.

\item[Timothy:]
\pdfbookmark[1]{Timothy}{Timothy}%
Modified the tables to 3NF or BCNF; created the descriptions of the tables.

\item[Schuyler:]
\pdfbookmark[1]{Schuyler}{Schuyler}%
Played a little catch\hyp up and wrote up these descriptions.

\end{enumerate}%
\end{samepage}

\end{document}
