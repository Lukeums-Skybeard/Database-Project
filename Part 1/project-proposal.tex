%!TEX encoding = UTF-8 Unicode
%!TEX TS-program = lualatexmk-shell
\RequirePackage{ucharcat,uspace,etex,shellesc}
\RequirePackage[1.7,compress,objcompress]{bxpdfver}
\documentclass[american,extrafontsizes,12pt,portrait,letterpaper,oneside,onecolumn,article,final]{memoir}
\settypeoutlayoutunit{in}
\pagestyle{empty}
\setulmarginsandblock{1.25in}{1.25in}{*}
\setlrmarginsandblock{1.25in}{1.25in}{*}
\setheadfoot{0.625in}{0.625in}
\usepackage{indentfirst,pdftexcmds,ragged2e,magicnum,atveryend,etextools,regexpatch,graphicx,dotlessi,checkend,calc,alphalph,thinsp,pbox,xfrac,stackengine,xsavebox}
\usepackage[table]{xcolor}
\usepackage{colorspace,longtable,multirow,bigstrut,bigdelim,fontaxes,arydshln,diagbox,collcell,scalerel,hhline}
\usepackage{amssymb,amsmath,amsthm,delim}
\usepackage{accents,mathtools,thmtools,empheq,iffont,unicode-math,realscripts,metalogo,mathpartir,luatextra}
\setmainfont{DejaVuSerif}[
Path={/Library/Fonts/},
Extension={.ttf},
Scale=MatchLowercase,
UprightFont={*},
BoldFont={*-Bold},
ItalicFont={*-Italic},
BoldItalicFont={*-BoldItalic},
FontFace={c}{n}{*Condensed},
FontFace={cb}{n}{*Condensed-Bold},
FontFace={c}{it}{*Condensed-Italic},
FontFace={cb}{it}{*Condensed-BoldItalic}
]
\setsansfont{DejaVuSans}[
Path={/Library/Fonts/},
Extension={.ttf},
Scale=MatchLowercase,
UprightFont={*},
BoldFont={*-Bold},
ItalicFont={*-Oblique},
BoldItalicFont={*-BoldOblique},
FontFace={c}{n}{*Condensed},
FontFace={cb}{n}{*Condensed-Bold},
FontFace={c}{it}{*Condensed-Oblique},
FontFace={cb}{it}{*Condensed-BoldOblique},
FontFace={el}{n}{*-ExtraLight}
]
\setmonofont{DejaVuSansMono}[
Path={/Library/Fonts/},
Extension={.ttf},
Scale=MatchLowercase,
UprightFont={*},
BoldFont={*-Bold},
ItalicFont={*-Oblique},
BoldItalicFont={*-BoldOblique}
]
\setmathfont{DejaVuMathTeXGyre}[
Path={/Library/Fonts/},
Extension={.ttf},
Scale=MatchLowercase
]
\setmathfont{DejaVuSerif}[
Path={/Library/Fonts/},
Extension={.ttf},
Scale=MatchLowercase,
range={up->up},
script-features={},
sscript-features={}
]
\setmathfont{DejaVuSerif-Bold}[
Path={/Library/Fonts/},
Extension={.ttf},
Scale=MatchLowercase,
range={bfup->up},
script-features={},
sscript-features={}
]
\setmathfont{DejaVuSerif-Italic}[
Path={/Library/Fonts/},
Extension={.ttf},
Scale=MatchLowercase,
range={it->up},
script-features={},
sscript-features={}
]
\setmathfont{DejaVuSerif-BoldItalic}[
Path={/Library/Fonts/},
Extension={.ttf},
Scale=MatchLowercase,
range={bfit->up},
script-features={},
sscript-features={}
]
\setmathfont{DejaVuSans}[
Path={/Library/Fonts/},
Extension={.ttf},
Scale=MatchLowercase,
range={sfup->up,bb->bb,bbit->bbit},
script-features={},
sscript-features={}
]
\setmathfont{DejaVuSans-Bold}[
Path={/Library/Fonts/},
Extension={.ttf},
Scale=MatchLowercase,
range={bfsfup->up},
script-features={},
sscript-features={}
]
\setmathfont{DejaVuSans-Oblique}[
Path={/Library/Fonts/},
Extension={.ttf},
Scale=MatchLowercase,
range={sfit->up},
script-features={},
sscript-features={}
]
\setmathfont{DejaVuSans-BoldOblique}[
Path={/Library/Fonts/},
Extension={.ttf},
Scale=MatchLowercase,
range={bfsfit->up},
script-features={},
sscript-features={}
]
\setmathfont{DejaVuSansMono}[
Path={/Library/Fonts/},
Extension={.ttf},
Scale=MatchLowercase,
range={tt->up},
script-features={},
sscript-features={}
]
\usepackage[inline]{enumitem}
\usepackage{linegoal}
\usepackage[linegoal,delarray]{tabu}
\usepackage{hyphenat,hyperref,polyglossia}
\setdefaultlanguage[variant=american]{english}
\usepackage[english=american,maxlevel=6]{csquotes}
\usepackage[american]{selnolig}
\usepackage[calc,useregional]{datetime2}
\usepackage{microtype,tikz,tikzpagenodes}
\usetikzlibrary{babel,scopes,quotes,calc,through,fit,matrix,chains,arrows.meta,circuits.logic.US,graphs,graphdrawing,automata,positioning,tikzmark}
\usegdlibrary{trees,layered,force,circular,routing}
\usepackage[tikz]{ocgx2}
%\usepackage{showhyphens}
\microtypesetup{activate=all}
\hypersetup{unicode}
\nolig{ffic}{f|fic}
\nolig{affin}{af|fin}
\tikzstyle{every picture}+=[remember picture,outer sep=auto]

\newcommand*\rnmath[1]{\(\displaystyle #1 \textstyle\)}
\newcommand\band{\mathbin{∧}}
\newcommand\bor{\mathbin{∨}}
\newcommand\bnot{\mathopen{∼}}
\newcommand\TT{\mathord{\symsfup{T}}}
\newcommand\FF{\mathord{\symsfup{F}}}
\newcommand\taut{\mathord{\symbfup{t}}}
\newcommand\cont{\mathord{\symbfup{c}}}
\newcommand\bqed{\mathpunct{∎}}
\newcommand\blank{\mathord{-}}
\newcommand\divslash{\mathbin{∕}}
\newcommand\fracslash{\mathbin{⁄}}
\DeclarePairedDelimiterX\divides[2]{}{}{#1 \mathrel{\delimsize∣} #2}
\DeclarePairedDelimiterX\ndivides[2]{}{}{#1 \mathrel{\delimsize∤} #2}
\newcommand\negative{\mathopen{-}}
\newcommand\mult{\mathbin{·}}
\newcommand\factorial{\mathclose{!}}
\DeclarePairedDelimiter\floor{⌊}{⌋}
\DeclarePairedDelimiter\ceiling{⌈}{⌉}
\newcommand\exto{\mathpunct{϶}}
\DeclarePairedDelimiter\card{\lvert}{\rvert}
\colorlet{gRed}{red!75!black}
\colorlet{gBlue}{blue!75!black}
\colorlet{gGray}{white!25!black}
\newcommand*\email[2][]{\href[#1]{mailto:#2}{\nolinkurl{#2}}}

\author{Timothy Gibson, Alexander Altman, and Schuyler Davis}
\title{Team GLASTA: Project Proposal}
\date{\DTMdate{2017-02-14}}

\begin{document}
\checkandfixthelayout[nearest]
\midsloppy
\begin{Center}
\textsf{\strong{\LARGE\thetitle}}
\end{Center}

\section*{Names}
\begin{Center}
\begin{tabu}[t]{rlr}
\strong{Project Name:} & \multicolumn{2}{c}{\textit{Fantastic Furniture}}\\
\tabucline[2\arrayrulewidth]{-}
\ldelim\{{3}{*}[\strong{Participants:}] & Timothy Gibson & \email{tgibson1@csustan.edu}\\
\tabucline{2-}
& Alexander Altman & \email{aaltman@csustan.edu}\\
\tabucline{2-}
& Schuyler Davis & \email{sdavis20@csustan.edu}\\
\end{tabu}
\end{Center}

\section*{Domain}
The domain of the database application will include different types of furniture.
These pieces of furniture will be divided into different categories and subsets of furniture.
These categories will then also have sub\hyp categories for further narrowing of customer search results.

\section*{User Group}
The intended user group for the database will be customers at a furniture store.
The database will allow customers to quickly find furniture that they are shopping for.
The database can be used by customers to see different types of furniture sold by the store and can sort them by different categories and characteristics such as color, size, type, and model, among others.

\section*{Modeling Scope}
The system will model a database of furniture for a furniture website.
The system will model different types of furniture, their categories, their different characteristics and the different applications of the furniture pieces.
The system will \emph{not} model an application that will be used by the store staff.
The system will \emph{not} include information that is important for employees who work at the store such as shipment dates, shipping routes for each piece, in stock dates, out of stock date, or other such data.
The intent of the database is to catalog all the information a customer seeking to buy furniture would find relevant.

\section*{Ground Rules}
\begin{samepage}
\begin{enumerate}[leftmargin=*]
\item Complete every task assigned to you.
\item Communicate with group members about problems or concerns.
\item Give as much feedback as possible.
\item Attend every group meeting.
\item Tell group members about inability to perform tasks or attend group meetings \emph{beforehand}.
\end{enumerate}
\end{samepage}

\section*{Possible Extensions}
The system could support a suggestion tool to suggest pieces of furniture to customers based on viewing or buying habits.
The system could also support a furniture listing system for customers to save lists of specific pieces of furniture for future viewing or purchase.
In addition, the system could enable the ability for users to search for items in the database based on different search terms, such as color, type, or SKU number, among others.

\end{document}
