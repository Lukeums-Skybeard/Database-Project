%!TEX encoding = UTF-8 Unicode
%!TEX TS-program = lualatexmk-shell
\RequirePackage{ucharcat,uspace,etex,shellesc}
\RequirePackage[1.7,compress,objcompress]{bxpdfver}
\documentclass[american,extrafontsizes,12pt,portrait,letterpaper,oneside,onecolumn,article,final]{memoir}
\settypeoutlayoutunit{in}
\pagestyle{plain}
\setulmarginsandblock{1in}{1in}{*}
\setlrmarginsandblock{1in}{1in}{*}
\setheadfoot{0.5in}{0.5in}
\usepackage{etoolbox,indentfirst,pdftexcmds,ragged2e,magicnum,atveryend,regexpatch,graphicx,dotlessi,checkend,calc,alphalph,thinsp,pbox,xfrac,xsavebox,xspace}
\usepackage[usestackEOL]{stackengine}
\usepackage[table]{xcolor}
\usepackage{colorspace,longtable}
\usepackage[longtable]{multirow}
\usepackage{bigstrut,bigdelim,fontaxes,arydshln,diagbox,collcell,scalerel,hhline,fvextra}
\usepackage{amssymb,amsmath,amsthm,delim}
\usepackage{accents,mathtools,thmtools,empheq,iffont,unicode-math,realscripts,metalogo,mathpartir,luatextra,minted}
\renewcommand\MintedPygmentize{/Library/Frameworks/Python.framework/Versions/Current/bin/pygmentize}
\usepackage[normalem]{ulem}
\setmainfont{DejaVuSerif}[
Path={/Library/Fonts/},
Extension={.ttf},
Scale=MatchLowercase,
UprightFont={*},
BoldFont={*-Bold},
ItalicFont={*-Italic},
BoldItalicFont={*-BoldItalic},
FontFace={c}{n}{*Condensed},
FontFace={cb}{n}{*Condensed-Bold},
FontFace={c}{it}{*Condensed-Italic},
FontFace={cb}{it}{*Condensed-BoldItalic}
]
\setsansfont{DejaVuSans}[
Path={/Library/Fonts/},
Extension={.ttf},
Scale=MatchLowercase,
UprightFont={*},
BoldFont={*-Bold},
ItalicFont={*-Oblique},
BoldItalicFont={*-BoldOblique},
FontFace={c}{n}{*Condensed},
FontFace={cb}{n}{*Condensed-Bold},
FontFace={c}{it}{*Condensed-Oblique},
FontFace={cb}{it}{*Condensed-BoldOblique},
FontFace={el}{n}{*-ExtraLight}
]
\setmonofont{DejaVuSansMono}[
Path={/Library/Fonts/},
Extension={.ttf},
Scale=MatchLowercase,
UprightFont={*},
BoldFont={*-Bold},
ItalicFont={*-Oblique},
BoldItalicFont={*-BoldOblique}
]
\setmathfont{DejaVuSerif}[
Path={/Library/Fonts/},
Extension={.ttf},
Scale=MatchLowercase,
range={up->up},
script-features={},
sscript-features={}
]
\setmathfont{DejaVuSerif-Bold}[
Path={/Library/Fonts/},
Extension={.ttf},
Scale=MatchLowercase,
range={bfup->up},
script-features={},
sscript-features={}
]
\setmathfont{DejaVuSerif-Italic}[
Path={/Library/Fonts/},
Extension={.ttf},
Scale=MatchLowercase,
range={it->up},
script-features={},
sscript-features={}
]
\setmathfont{DejaVuSerif-BoldItalic}[
Path={/Library/Fonts/},
Extension={.ttf},
Scale=MatchLowercase,
range={bfit->up},
script-features={},
sscript-features={}
]
\setmathfont{DejaVuSans}[
Path={/Library/Fonts/},
Extension={.ttf},
Scale=MatchLowercase,
range={sfup->up,bb->bb,bbit->bbit},
script-features={},
sscript-features={}
]
\setmathfont{DejaVuSans-Bold}[
Path={/Library/Fonts/},
Extension={.ttf},
Scale=MatchLowercase,
range={bfsfup->up},
script-features={},
sscript-features={}
]
\setmathfont{DejaVuSans-Oblique}[
Path={/Library/Fonts/},
Extension={.ttf},
Scale=MatchLowercase,
range={sfit->up},
script-features={},
sscript-features={}
]
\setmathfont{DejaVuSans-BoldOblique}[
Path={/Library/Fonts/},
Extension={.ttf},
Scale=MatchLowercase,
range={bfsfit->up},
script-features={},
sscript-features={}
]
\setmathfont{DejaVuSansMono}[
Path={/Library/Fonts/},
Extension={.ttf},
Scale=MatchLowercase,
range={tt->up},
script-features={},
sscript-features={}
]
\setmathfont{DejaVuMathTeXGyre}[
Path={/Library/Fonts/},
Extension={.ttf},
Scale=MatchLowercase
]
\usepackage[inline]{enumitem}
\usepackage{linegoal}
\usepackage[linegoal,delarray]{tabu}
\usepackage{hyphenat,hyperxmp,hyperref,polyglossia}
\setdefaultlanguage[variant=american]{english}
\usepackage[english=american,maxlevel=6]{csquotes}
\usepackage[american]{selnolig}
\usepackage[calc,useregional]{datetime2}
\usepackage{microtype,tikz,tikzpagenodes}
\usetikzlibrary{babel,scopes,quotes,calc,through,fit,matrix,chains,arrows.meta,circuits.logic.US,graphs,graphdrawing,automata,positioning,er,tikzmark}
\usegdlibrary{trees,layered,force,circular,routing}
\usepackage[tikz]{ocgx2}
%\usepackage{showhyphens}
\microtypesetup{activate=all}
\hypersetup{unicode=true,pdflang={en}}
\usemintedstyle{xcode}
\nolig{ffic}{f|fic}
\nolig{affin}{af|fin}
\tikzstyle{every picture}+=[remember picture,outer sep=auto]

\newcommand*\rnmath[1]{\(\displaystyle #1 \textstyle\)}
\newcommand\band{\mathbin{∧}}
\newcommand\bor{\mathbin{∨}}
\newcommand\bnot{\mathopen{∼}}
\newcommand\TT{\mathord{\symsfup{T}}}
\newcommand\FF{\mathord{\symsfup{F}}}
\newcommand\taut{\mathord{\symbfup{t}}}
\newcommand\cont{\mathord{\symbfup{c}}}
\newcommand\bqed{\mathpunct{∎}}
\newcommand\blank{\mathord{-}}
\newcommand\divslash{\mathbin{∕}}
\newcommand\fracslash{\mathbin{⁄}}
\DeclarePairedDelimiterX\divides[2]{}{}{#1 \mathrel{\delimsize∣} #2}
\DeclarePairedDelimiterX\ndivides[2]{}{}{#1 \mathrel{\delimsize∤} #2}
\newcommand\negative{\mathopen{-}}
\newcommand\mult{\mathbin{·}}
\newcommand\factorial{\mathclose{!}}
\DeclarePairedDelimiter\floor{⌊}{⌋}
\DeclarePairedDelimiter\ceiling{⌈}{⌉}
\newcommand\exto{\mathpunct{϶}}
\DeclarePairedDelimiter\card{\lvert}{\rvert}
\newcommand*\rproj[1]{\mathop{π_{#1}}}
\newcommand*\rsel[1]{\mathop{σ_{#1}}}
\newcommand*\rname[1]{\mathop{ρ_{#1}}}
\newcommand*\rjoin[1]{\mathbin{⋈_{#1}}}
\newcommand\rcross{\mathbin{×}}
\newcommand\runion{\mathbin{∪}}
\newcommand\rintersection{\mathbin{∩}}
\newcommand\rdiff{\mathbin{-}}
\colorlet{gRed}{red!75!black}
\colorlet{gBlue}{blue!75!black}
\colorlet{gGray}{white!25!black}
\newcommand*\email[2][]{\href[#1]{mailto:#2}{\nolinkurl{#2}}}
\newcommand*\TODO{\textcolor{red}{[TODO]}\xspace}
\newcommand*{\sqli}[1]{\mintinline[breaklines,breakbytokenanywhere]{mysql}{#1}}
\newcommand*{\sqlm}[1]{\text{\sqli{#1}}}
\newcommand*{\mysqli}[1]{\mintinline[breaklines,breakbytokenanywhere]{mysql}{#1}}
\newcommand*{\mysqlm}[1]{\text{\mysqli{#1}}}

\author{Timothy Gibson, Alexander Altman, and Schuyler Davis}
\title{Team GLASTA's \emph{Fantastic Furniture}: Database Queries}
\date{\DTMdate{2017-05-02}}

\begin{document}
\checkandfixthelayout[nearest]
\midsloppy
\begin{Center}
\textsf{\strong{\LARGE\thetitle}}
\end{Center}

\begin{samepage}%
\section*{Team Info}
\pdfbookmark[0]{Team Info}{team-info}%
\begin{Center}
\begin{tabu}[t]{rlr}
\strong{Team Name:} & \multicolumn{2}{c}{Team GLASTA}\\
\tabucline[2\arrayrulewidth]{-}
\strong{Project Name:} & \multicolumn{2}{c}{\emph{Fantastic Furniture}}\\
\tabucline[2\arrayrulewidth]{-}
\ldelim\{{3}{*}[\strong{Participants:}] & Timothy Gibson & \email{tgibson1@csustan.edu}\\
\tabucline{2-}
& Alexander Altman & \email{aaltman@csustan.edu}\\
\tabucline{2-}
& Schuyler Davis & \email{sdavis20@csustan.edu}\\
\end{tabu}
\end{Center}%
\end{samepage}

\section*{SQL Schemas}
\pdfbookmark[0]{SQL Schemas}{schemas}%

\inputminted[linenos,breaklines,breakbytokenanywhere]{mysql}{mysql.sql}

\section*{Sample Queries}
\pdfbookmark[0]{Sample Queries}{queries}%
\begin{enumerate}[leftmargin=*,label={\strong{\uline{(\emph{\Alph*})}}}]

\item
\begin{enumerate}[leftmargin=*,widest={\strong{Explanation:}}]
\item[\strong{Intent:}] \textquote[][?]{How many models has each designer designed}
\item[\strong{Query:}]
\begin{minted}[breaklines,breakanywhere]{mysql}
SELECT M.designerID,
       COUNT(C.modelNumber)
FROM   Designer D
       LEFT OUTER JOIN make M
                    ON ( D.designerID = M.designerID )
       INNER JOIN contains_ C
               ON ( M.setID = C.setID )
GROUP  BY M.designerID;
\end{minted}
\item[\strong{Result:}]
\begin{minted}[breaklines,breakanywhere]{text}
+------------+----------------------+
| designerID | count(C.modelNumber) |
+------------+----------------------+
| 3lejckzNYS |                   69 |
| 4MVbu2iI15 |                  170 |
| 7RYpyw9es0 |                  212 |
| 7ZZbCsXnv0 |                  210 |
| A7oA1v9Ax1 |                  212 |
| AZadqlHsUN |                  232 |
| b9yOGUx3pl |                  213 |
| bBMMbiXWX2 |                  189 |
| bDcoTRYgku |                  148 |
| BejwYSNzm7 |                  142 |
| bjSmt6EX8o |                  193 |
| bnQDB9V4ZQ |                  215 |
| bOQ84LG8yQ |                  246 |
| BVP4o4g0u6 |                  121 |
| c1jIajAyha |                  142 |
| czfBYIFNhs |                  252 |
| eAm5FaKjru |                  196 |
| ejSgB4P19T |                  215 |
| EZn2c6Sqao |                  164 |
| fLM0vFMD6h |                  190 |
| HACGEeYiTg |                  182 |
| hL6koxT8vK |                  265 |
| HULdxdPYgo |                  130 |
| HUPang5JW4 |                  181 |
| in19yTwFqy |                  254 |
| jMt9cpvHJ8 |                  218 |
| kFNSGfDIXN |                  200 |
| KH4hmznKQN |                  252 |
| KoaWPsykpt |                  244 |
| L2vFnWn5yt |                  170 |
| lA2l4dUPAN |                  265 |
| MBxLmTyDx0 |                  152 |
| nUFwyC0BAj |                  167 |
| nxtwKjphvt |                  260 |
| OLkEycvtOv |                   99 |
| p3f6twELII |                  203 |
| P9Vg4XQ5AK |                  267 |
| pIdz2ArSJd |                  168 |
| PWFixIVSN0 |                  197 |
| ql0bQqFgKx |                  224 |
| rdYk2JSFOZ |                  229 |
| sUUNXUZjSB |                  174 |
| TnFL7eVZD9 |                  246 |
| UKHmNCJ1Ep |                  211 |
| uQJBa7yRRm |                  240 |
| VXBvjILW4l |                  255 |
| VyLXDToji5 |                  186 |
| Xi0fOUOila |                  200 |
| YOpJVyms0T |                  151 |
| yz0xcnsOMA |                  212 |
+------------+----------------------+
50 rows in set (0.01 sec)
\end{minted}
\item[\strong{Explanation:}] This query is plausible because someone might want to check which designers were more prolific; this might inform their decision on whose furniture to buy.

This result is sensible, though somewhat unlikely in the real world; there are 50 designers and 1000 models, and each designer has designed about 200 models on average, so each model has about 10 designers.
Further investigation into the data validated this calculation, so the query gave the intended result, even if that result was unintuitive.
\end{enumerate}

\item
\begin{enumerate}[leftmargin=*,widest={\strong{Explanation:}}]
\item[\strong{Intent:}] \textquote[][?]{What is the average lead time from suppliers to distribution centers when both are in the same country}
\item[\strong{Query:}]
\begin{minted}[breaklines,breakanywhere]{mysql}
SELECT AVG(leadTime)
FROM   Supplier S,
       DistributionCenter C,
       canOrderFrom O
WHERE  S.supplierID = O.supplierID
   AND C.centerID = O.centerID
   AND S.country = C.country;
\end{minted}
\item[\strong{Result:}]
\begin{minted}[breaklines,breakanywhere]{text}
+-------------------+
| AVG(leadTime)     |
+-------------------+
| 5.333333333333333 |
+-------------------+
1 row in set (0.00 sec)
\end{minted}
\item[\strong{Explanation:}] This query is plausible because someone might want to know how long, on average, they'd have to wait for furniture to arrive at the distribution centers after being ordered.

This result is sensible because it gives a single result to a query that contains only an aggregation function and no grouping.
\end{enumerate}

\item
\begin{enumerate}[leftmargin=*,widest={\strong{Explanation:}}]
\item[\strong{Intent:}] \textquote[][?]{Which suppliers can get me a claw\hyp footed bedframe in less than a week}
\item[\strong{Query:}]
\begin{minted}[breaklines,breakanywhere]{mysql}
SELECT DISTINCT O.supplierID
FROM   Bedframe B,
       describes D,
       features_Feature F,
       canOrderFrom O,
       stocks S
WHERE  B.sku = D.sku
   AND D.modelNumber = F.modelNumber
   AND F.description = 'Claw Feet'
   AND O.centerID = S.centerID
   AND S.sku = B.sku
   AND O.leadTime < 7.0;
\end{minted}
\item[\strong{Result:}]
\begin{minted}[breaklines,breakanywhere]{text}
+------------+
| supplierID |
+------------+
| 45AVHG6SDL |
| dWuUGFOe1b |
| lGSxEQQ8bN |
| 4hGbj2aBVR |
| socFXUsXAD |
| 0x1V8UVLqh |
| umALkT56Xy |
| xUMVYgBI7J |
| FE34LTqb2p |
| ldbBahbqLj |
| ryNb8O1TRs |
| tRJBEnEFaO |
| ATeybFIuTQ |
| 7n94AZBOkB |
| Op8rarVKxa |
| FyAzhOpNly |
| WY3AmkQmxs |
| xyEInQvE1U |
| ZxsJeaFpg5 |
| 06DeQdaf4X |
| 5SKHZGJsjV |
| FVzpxLJFKK |
| uluou8izCd |
| nZ4msLXxTY |
| x7FolH1EsA |
| xJZiBD5DIo |
| cUrmX3GV4D |
| KQIGJ9eDk5 |
| 8pYnVWzpzR |
| 9Hzdyajzdu |
| 9kQZDUILDP |
| 3LnmekGoPa |
| 4xp0ETTkc8 |
| IBvWjkFney |
| XIFO3VhtQH |
+------------+
35 rows in set (0.00 sec)
\end{minted}
\item[\strong{Explanation:}] This query is plausible because a customer might very well want a bedframe---with particular attributes, even---within a specified time limit.

This result seems reasonable, because it feels fairly realistic that claw\hyp footed bedframes wouldn't be particularly rare, so 35 of our 50 available suppliers carrying them seems fine.
It should be noted that, the first time this query was run, we forgot the \mysqli{DISTINCT} keyword after \mysqli{SELECT}; this gave us 100 (obviously non\hyp unique) results, and taught us a lesson about not assuming automatic distinctness in the result of a complex query just because the result column is a primary key in its original table.
\end{enumerate}

\item
\begin{enumerate}[leftmargin=*,widest={\strong{Explanation:}}]
\item[\strong{Intent:}] \textquote[][?]{How many items is my old buddy Harold stocking in that warehouse of his}
\item[\strong{Query:}]
\begin{minted}[breaklines,breakanywhere]{mysql}
SELECT COUNT(*)
FROM   stocks S,
       DistributionCenter D
WHERE  S.centerID = D.centerID
   AND D.name_ = 'Harold and His Big Ass Warehouse';
\end{minted}
\item[\strong{Result:}]
\begin{minted}[breaklines,breakanywhere]{text}
+----------+
| COUNT(*) |
+----------+
|        7 |
+----------+
1 row in set (0.00 sec)
\end{minted}
\item[\strong{Explanation:}] This query is plausible because somebody might actually care about Harold's business success (or lack thereof).
We wanted to make a query that involved a specific name or attribute to make sure we could return small data as well as larger data.

Apparently Harold has a lot of wasted space right now; blame the economy.
\end{enumerate}

\item
\begin{enumerate}[leftmargin=*,widest={\strong{Explanation:}}]
\item[\strong{Intent:}] \textquote[][?]{Which sets have more than two chairs and at least one table}
\item[\strong{Query:}]
\begin{minted}[breaklines,breakanywhere]{mysql}
SELECT DISTINCT C.setID
FROM   contains_ C
WHERE  (SELECT COUNT(DISTINCT I.sku) * C.count_
        FROM   Table_ I,
               describes D
        WHERE  D.sku = I.sku
           AND D.modelNumber = C.modelNumber) >= 1
   AND (SELECT COUNT(DISTINCT I.sku) * C.count_
        FROM   Chair I,
               describes D
        WHERE  D.sku = I.sku
           AND D.modelNumber = C.modelNumber) > 2;
\end{minted}
\item[\strong{Result:}]
\begin{minted}[breaklines,breakanywhere]{text}
Empty set (0.00 sec)
\end{minted}
\item[\strong{Explanation:}] This query is plausible because someone might want some dinner furniture in a professionally matched set.

I guess we don't carry any of that kind of set; oh well!
\end{enumerate}

\end{enumerate}

\section*{Group Work}
\pdfbookmark[0]{Group Work}{group-work}%
\begin{samepage}%
\begin{enumerate}[leftmargin=*,widest={Alexander:}]

\item[Timothy:]
\pdfbookmark[1]{Timothy}{Timothy}%
Checked the queries and results for sensibility.

\item[Alexander:]
\pdfbookmark[1]{Alexander}{Alexander}%
Ran the queries against the database and recorded their results.

\item[Schuyler:]
\pdfbookmark[1]{Schuyler}{Schuyler}%
Checked the queries and results for sensibility.

\end{enumerate}%
\end{samepage}

\end{document}
